\section{Aufgabe 1}
Der Kompressionsmodul ist durch $K = -\frac{\mathrm{d}p \cdot V}{\mathrm{d}V} $ definiert. 
\begin{align*}
    \lambda_{Wasser} = \frac{c}{f} = \frac{1492}{2.5 \cdot 10^6} = 5.968 \cdot 10^{-4} m \\
    \lambda_{Muskel} = \frac{c}{f} = \frac{1568}{2.5 \cdot 10^6} = 6.272 \cdot 10^{-4} m \\
    R = (\frac{Z_2 - Z_1}{Z_2 + Z_1})^2 = (\frac{1.48 - 1.63}{1.48 + 1.63})^2 = 0.413\%
\end{align*}


\section{Aufgabe 2}
\begin{itemize}
    \item Reflektion:
    Erst durch das reflektierte Signal kann überhaupt eine Bildgebung stattfinden. Da die Reflektion an den Übergängen zweier unterschiedlicher Gewebe gut gemessen werden kann, kann man Aussagen darüber treffen wo sich diese Übergänge befinden. 
    \item Dämpfung: Durch unterschiedliche Abschwächungen von Ultraschallwellen in Geweben kann man feststellen, um welche Art von Gewebe es sich handelt.
\end{itemize}

Zwei Wechselwirkungen, die sich indirekt auf die Dämpfung auswirken:
\begin{itemize}
    \item Refraktion
    \item Streuung
\end{itemize}


\section{Aufgabe 3}
\begin{align*}
    A-mode: PRF &= \frac{c}{x_{max}} = \frac{1540}{2 \cdot 10 \cdot 10^{-2}} = 7.7 \cdot 10^3 \\
    B-mode: FR &= \frac{PRF}{Linien} = \frac{7.7 \cdot 10^3}{256} = 30.07
\end{align*}

\section{Aufgabe 4}

\begin{align*}
    Frequenzverschiebung: \Delta f = 2 \frac{v \cdot f}{c} \cdot \cos(\theta) = 4870.13Hz \\
    PRF = \frac{4 \cdot f \cdot v_{max} \cdot cos(\theta)}{c} = 9740.26Hz
\end{align*}

Nach dem Nyquist-Abtasttheorem kann das Signal exakt rekonstruiert werden, wenn die Abtastrate mindestens doppelt so groß ist als die Frequenzverschiebung. 
Demnach kann die Flussgeschwindigkeit von 1,5m/s detektiert werden.
%\section{Aufgabe 5}
