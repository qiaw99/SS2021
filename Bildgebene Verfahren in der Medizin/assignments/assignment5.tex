\section{Aufgabe 1}
Neutronenüberschuss. \\
Reaktionsgleichung: \isotope[2][]{H} + \isotope[14][]{N} $\xrightarrow{3 MeV}$ \isotope[15][]{O} + n   \newline

Ein instabiler Atomkern mit einem Überschuss an Neutronen kann einem $\beta$-Zerfall unterliegen, bei dem ein Neutron in ein Proton, ein Elektron und ein Elektron-Antineutrino umgewandelt wird:
\begin{align*}
    n \longrightarrow p + \beta^+ + \overline{V_e}
\end{align*}
\section{Aufgabe 2}
Reaktionsgleichung: \isotope[11][6]{C} $\longrightarrow$ \isotope[11][5]{B} + $\beta^+$ \newline \newline
\textbf{Vorteile: }
\begin{itemize}
    \item Die Moleküle sind wenig radioaktiv, denn sie haben normalerweise eine niedrigenergetische Gammastrahlung.
    \item Die Markierungszeit soll sehr kurz gehalten werden, denn wenige Halbwertszeiten Synthesezeit erfordern.
\end{itemize}

\textbf{Nachteile:}
\begin{itemize}
    \item Die Isotopen mit längeren Halbwertszeiten können unter Aussendung von Positronen mit der niedrigsten Positronenenergie zerfallen, was zu einer hochauflösenden Bildaufnahme beiträgt.
    \item Wegen der kurzen Halbwertszeit müssen die Produktion und der Einbau der Radionuklide in Moleküle am Anwendungsort stattfinden, wozu man lokal ein Zyklotron braucht. 
\end{itemize}

\section{Aufgabe 3}
Durch den Zerfall eines Elements entsteht es ein Positron, was mit einer Elektron zur Elektron-Positron-Annihilation führt. Während dieses Vorgang erzeugt ein Paar von Annihilationsphotonen(Röntgenstrahlung), die in zwei Richtungen fast Rücken an Rücken geschossen werden.

\section{Aufgabe 4}
Röntgenquanten besitzen eine erheblich größere Energie als sichtbares Licht. Sie können Stoffe(z.B. Kristallgitter) ionisieren.

\section{Aufgabe 5}
\begin{itemize}
    \item Koinzidenz durch Streuung: Mithilfe von der Streuung wegen der Gewerbe kommt es zu Koinzidenz. Während die zwei Photonen vorher nicht die gegengesetzte Richtung haben. 
    \item Zufällig Koinzidenz: Es gibt zwei Zerstrahlungen der Positronium, die beiden nicht zu Koinzidenz führen. Aber jeweils von beiden führen zu Koinzidenz.
    \item Merhfachkoinzidenz: Es gibt zwei Zerstrahlungen der Positronium, die jeweils zu Koinzidenz führen.
\end{itemize}