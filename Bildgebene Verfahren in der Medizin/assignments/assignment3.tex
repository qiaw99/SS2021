\section{Aufgabe 1}
\textbf{Neutronen} haben größeren Schaden. Denn Neutronradiation ist in manchen Fällen problematisch: Neutronenstrahlung hat die Fähigkeit, in den Körpergeweben, auf die sie trifft, Radioaktivität zu induzieren. Dies geschieht durch die Aufnahme von Neutronen durch den Kern des Atoms, das in ein anderes Nuklid umgewandelt wird.

\section{Aufgabe 2}
\begin{itemize}
    \item \textbf{Komponente einer Röntgenröhre: }
        \begin{itemize}
            \item[(1)] Kathode (Glühwendel mit Zylinder)
            \item[(2)] Anode / Anodenteller
            \item[(3)] Glasröhre
            \item[(4)] Rotor
        \end{itemize}
    \item \textbf{Entstehung der Bremsstrahlung: } Die vorbeifliegenden Elekronen verlieren ihre kinetischen Energien durch die Coulomb Anziehung von dem Atomkern. Die verlorenen kinetischen Energien werden dann in Form von Röntgenstrahlung (und Wärme) umgewandelt.
\end{itemize}
\section{Aufgabe 3}

Die Röntgenstrahlen werden auf ihrem Weg zum Detektor durch das im Weg befindliche Gewebe unterschiedlich abgeschwächt (nach dem Schwächungsgesetz). Die Detektoren nehmen ein Dosisprofil wahr, welches daraufhin durch ein geignetes Verfahren in ein für das menschliche Auge sichtbares Röntgenbild umgewandelt wird.

Ein solches Verfahren ist zum Beispiel die Schwarzweißphotographie. Dabei verwendet man eine in Gelatine eingebettet Schicht des lichtempfindlichen Materials Silberhalogenid. Dieses wird dort zu Silber reduziert, wo das Licht eintrifft. Dadurch entstehen helle oder dunkle Stellen im Bild.

\section{Aufgabe 4}
$\lambda_1 = \frac{h\cdot c}{E} = \frac{1.24}{80} = 0.0155nm$ \\
$\lambda_2 = \frac{h\cdot c}{E} = \frac{1.24}{25} = 0.0496nm$

\section{Aufgabe 5}
BP = 6

$\rightarrow$ $0.75^6 * 2.8mGy = 0.49834 mGy$\\

Dies entspricht einer Verringerung der Dosis um $82.2\%$.
